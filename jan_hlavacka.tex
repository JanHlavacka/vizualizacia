\documentclass[11pt, a4paper]{article}

\usepackage[utf8x]{inputenc}
\usepackage[slovak]{babel}
\usepackage{amsmath}
\usepackage{hyperref}
\usepackage{graphicx}
\usepackage{float}
\author{Jan Hlavacka}
\title{Zadanie v \LaTeX u}

\begin{document}

\maketitle
\newpage
\begin{abstract}
    Zadanie z predmetu Vizualizácia. Prvý krok pozostáva z vygenerovania
    súboru .vtk, ktorý prezentuje geometrický objekt torus. V druhom kroku
    otvoríme súbor v Paraview a vyskúšame rôzne zobrazenia a filtre. Na zátver
    to spracujeme v Latex.
\end{abstract}
\newpage
\tableofcontents
\newpage
\section{Vygenerovanie torusu}
Program musí na začiatok vygenerovať "hlavičku", podľa ktorej ParaView rozpozná, že ide o súbor .vtk. Hlavička musí obsahovať následné informácie.
\begin{enumerate}
    \item vtk DataFile Version 2.0 - identifikácia verzie VTK
    \item nadpis
    \item dátový typ súboru - ASCII, alebo BINARY
    \item dátová štruktúra - definuje geometriu a topológiu
    \item vlastnosti dát
\end{enumerate}
Ďalší krok je vygenerovať body Torusu. Zadáme si počet horizontálnych a vertikálnych delení.
Body torusu získame z parametrických rovníc:

\begin{center}
    \begin{eqnarray*}
        x(\theta,\varphi) &=& (R + r cos\theta)cos\varphi \\
        y(\theta,\varphi) &=& (R + r cos\theta)sin\varphi \\
        z(\theta,\varphi) &=& r sin\theta \\
    \end{eqnarray*}
\end{center}

Nakoniec zapíšeme jednotlivé polygóny. V našom prípade trojuholníky, ktoré sú navzájom opačne orientované.
\newpage
\section{obrazové výstupy}
\subsection[]{Zobrazenie Torusu}
\begin{figure}[H]
    \includegraphics[width=0.6\textwidth]{points_no_gaussian.png}
    \includegraphics[width=0.6\textwidth]{points.png}
    \caption{Torus, zobrazený ako body. Napravo Gaussian}
\end{figure}
\begin{figure}[H]
    \includegraphics[width=0.6\textwidth]{wireframe.png}
    \includegraphics[width=0.6\textwidth]{wireframe_6_4.png}
    \caption{Torus, zobrazený ako wireframe}
\end{figure}
\begin{figure}[H]
    \includegraphics[width=0.6\textwidth]{surface.png}
    \includegraphics[width=0.6\textwidth]{plochy_6_4.png}
    \caption{Torus, zobrazený ako wireframe}
\end{figure}
\begin{figure}[H]
    \includegraphics[width=0.6\textwidth]{wireframe.png}
    \includegraphics[width=0.6\textwidth]{wireframe_6_4.png}
    \caption{Torus, zobrazený ako plochy}
\end{figure}
\begin{figure}[H]
    \centering
    \includegraphics[width=1\textwidth]{surface_with_edge.png}
    \caption{Torus, zobrazený ako plocha s drôtenou reprezentáciou}
\end{figure}
\subsection[]{Aplikácia filtrov}
\begin{figure}[H]
    \includegraphics[width=0.6\textwidth]{clip_vertical_2.png}
    \includegraphics[width=0.6\textwidth]{clip_vertical_3.png}
    \caption{Vertikálny rez objektom}
\end{figure}
\begin{figure}[H]
    \centering
    \includegraphics[width=1\textwidth]{clip_horizont.png}
    \caption{Horizontálny rez objektom}
\end{figure}
\begin{figure}[H]
    \includegraphics[width=0.6\textwidth]{vysek.png}
    \includegraphics[width=0.6\textwidth]{vysek2.png}
    \caption{Kombinácia rezov - výseky}
\end{figure}

\newpage
\section{zdroje}
\begin{thebibliography}{3}
    \bibitem{latex}
    Akila Maithripala , \textit{Latex}
    \url{https://dev.to/ucscmozilla/how-to-create-and-compile-latex-documents-on-visual-studio-code-3jbk?fbclid=IwAR1jawbAVrN3GXvUvce7K9ARr7XMlHDfVLYHXPyRPkiUjmXCtYb-e5nKLpE}
    \bibitem{torus}
    Autor neznamy , \textit{Wikipedia}.
    \url{https://en.wikipedia.org/wiki/Torus}
    \bibitem{Paraview}
    Autor neznamy , \textit{tutorials}.
    \url{https://docs.paraview.org/en/latest/Tutorials/SelfDirectedTutorial/basicUsage.html#filters}

\end{thebibliography}
\end{document}